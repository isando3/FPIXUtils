% ----------------------------------------------------------------------

\subsection{\iv Test}
\label{ss:iv}

\subsubsection{Purpose}

The \iv test is unique in that it doesn't directly utilize the read-out functionality of the \roc.
In fact, this test can be performed on a sensor before it's even bump-bonded onto a set of \rocs.
In principle, the silicon sensor acts like a diode, 
allowing current to flow in the ``forward'' direction but not in the ``reverse'' direction.
A potential difference is created across the sensor, in the plane transverse to the sensor face,
by applying a ``reverse bias'' voltage.
This potential difference is what draws the electrons created by a charged particle passing through the sensor 
towards the bump-bond to be collected.
For a given sensor, the reverse bias has an operating range.
If the bias is too small, the sensor will not be depleted of free electrons and will not operate properly.
If the bias is too large, the sensor will break down and start acting like a resistor instead of a diode.
The \iv test is meant to determine the operating range of the sensor, 
bounded by the depletion voltage (lower limit) and the breakdown voltage (upper limit).

\subsubsection{\textcolor{red}{Methodology}}
\subsubsection{\textcolor{red}{Output}}

